\documentclass[conference]{IEEEtran}
\IEEEoverridecommandlockouts
% The preceding line is only needed to identify funding in the first footnote. If that is unneeded, please comment it out.
\usepackage{cite}
\usepackage{amsmath,amssymb,amsfonts}
\usepackage{algorithmic}
\usepackage{graphicx}
\usepackage{textcomp}
\usepackage{xcolor}
\def\BibTeX{{\rm B\kern-.05em{\sc i\kern-.025em b}\kern-.08em
    T\kern-.1667em\lower.7ex\hbox{E}\kern-.125emX}}
\begin{document}

\title{Ellipsoid Algorithm }

\author{\IEEEauthorblockN{Siddharth Bhat}
\IEEEauthorblockA{\textit{20161105}} \\
% \textit{name of organization (of Aff.)}\\
% City, Country \\
% email address}
%\and
%\IEEEauthorblockN{2\textsuperscript{nd} Given Name Surname}
%\IEEEauthorblockA{\textit{dept. name of organization (of Aff.)} \\
%\textit{name of organization (of Aff.)}\\
%City, Country \\
%email address}
%\and
%\IEEEauthorblockN{3\textsuperscript{rd} Given Name Surname}
%\IEEEauthorblockA{\textit{dept. name of organization (of Aff.)} \\
%\textit{name of organization (of Aff.)}\\
%City, Country \\
%email address}
%\and
%\IEEEauthorblockN{4\textsuperscript{th} Given Name Surname}
%\IEEEauthorblockA{\textit{dept. name of organization (of Aff.)} \\
%\textit{name of organization (of Aff.)}\\
%City, Country \\
%email address}
%\and
%\IEEEauthorblockN{5\textsuperscript{th} Given Name Surname}
%\IEEEauthorblockA{\textit{dept. name of organization (of Aff.)} \\
%\textit{name of organization (of Aff.)}\\
%City, Country \\
%email address}
%\and
%\IEEEauthorblockN{6\textsuperscript{th} Given Name Surname}
%\IEEEauthorblockA{\textit{dept. name of organization (of Aff.)} \\
%\textit{name of organization (of Aff.)}\\
%City, Country \\
%email address}
}

\newcommand{\PTIME}{\texttt{PTIME}}
\maketitle

\begin{abstract}
We motivate the Ellipsoid algorithm, discuss its original theoretical
importance, and remark on its practical efficiency.
\end{abstract}

\begin{IEEEkeywords}
component, formatting, style, styling, insert
\end{IEEEkeywords}

\section{Introduction}
The ellipsoid algorithm was discovered in Naum Z. Shor. Later, 
Leonid Genrikhovich Khachiyan proved that the algorithm runs in polynomial
time. This was a breakthrough in the theory of linear programming, which
proved that solving LP's is in \PTIME.

\section{Description the algorithm: Checking for non-emptiness}

\section{Using non-emptiness to solve LP's}
So far, all we can do using the ellipsoid algorithm is to check if some
system of equations $Ax \leq b$ is \textit{non-empty}. In other words, we
can check the non-emptiness of a given polyhedron. Here, we will describe how
to use this to solve \textit{optimisation problems}.

Consider a linear program and its dual:
\begin{align*}
        &x, c \in \mathbb R^{n \times 1} \quad A \in \mathbb R^{m \times n} \quad b, y \in \mathbb R^{m \times 1}\\
        &P_{primal} \equiv \underset{x}{\text{maximise }} c^T x \text{ subject to } Ax = b \\
        &P_{dual} \equiv \underset{y}{\text{minimise }} b^T y \text{ subject to } A^Ty \geq c \quad
\end{align*}

Let $x^\star$ be the optimal value of $x$ for $P_{primal}$, and $y^\star$ be the
optimal value of $y$ for $P_{dual}$. From strong duality, we know that
the value of $c^Tx^\star = b^Ty^\star$.

So, we can create a \textit{combined} linear program, whose feasibility will
force us to provide a point such that $c^Tx = b^T y$. That is, we create
a new polyhedra $Q$ defined by the equations:

\begin{align*}
        A x \leq b \quad
        A^T y \geq c \quad
        c^Tx = b^T y
\end{align*}

Now, if a feasible point $(x_0, y_0) \in Q$, then it must be the case that $Ax_0 = b$,
$A^Ty_0 \geq c$, and $c^Tx_0 = b^Ty_0$. At this point, strong duality tells us that
$(x_0, y_0) = (x^\star, y^\star)$. 

Hence, we can find the optimal value of the linear program by evaluating $c^T x_0$.


Thus, the ellipsoid algorithm can be used to solve for the optimality of a linear
program, by starting from a non-emptiness check! This is beautiful, and proves
a deep result of LP's: A certificate of non-emptiness is as good as a ceritificate
of optimality.

\bibliographystyle{IEEEtran}
\bibliography{references.bib}

\end{document}
